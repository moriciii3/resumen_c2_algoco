% dp/coin_change_min.tex

\subsubsection{Cantidad Mínima de Monedas (Coin Change Minimum)}

El problema busca encontrar el número mínimo de monedas requeridas para alcanzar un monto objetivo $X$, dado un conjunto de denominaciones $C_j$ y suministro ilimitado.

\paragraph{Fórmula DP:}
$$
DP[i] = 1 + \min_{j} \left\{ DP[i - C_j] \right\} \quad \text{si } i \ge C_j
$$

\paragraph{Complejidad:}
* \textbf{Tiempo:} $O(N \cdot X)$, donde $N$ es el número de monedas y $X$ el monto objetivo.
* \textbf{Espacio:} $O(X)$.

% --- Código C++ ---
\subsubsection{Código C++ (Cantidad Mínima de Monedas)}
\begin{lstlisting}[
	caption={Coin Change: Mínimo número de monedas (DP)},
	extendedchars=true, 
	literate={ñ}{{\~n}}1 {Ñ}{{\~N}}1 {á}{{\'a}}1 {é}{{\'e}}1 {í}{{\'i}}1 {ó}{{\'o'}}1 {ú}{{\'u}}1 {Á}{{\'A}}1 {É}{{\'E}}1 {Í}{{\'I}}1 {Ó}{{\'O}}1 {Ú}{{\'U}}1
	]
	#include <bits/stdc++.h>
	using namespace std;
	11 inf = 1e9; // Definición aproximada en el contexto del PDF
	using 11 long long;
	
	
	int main() {
		11 n, x;
		cin >> n >> x;
		
		vector<11> dp(x + 1, inf);
		vector<11> coins (n);
		
		for (ll i=0; i < n; i++) {
			cin >> coins [i];
		}
		
		dp [0] = 0;
		
		for (11 i=1; i <= x; i++) { // Corregí el <= X que faltaba en el bucle del PDF
			for (11 j=0; j<n; j++)
			if (i >= coins [j])
			{
				dp[i] = min (dp[i], dp[i - coins [j]] + 1);
			}
		}
		
		if (dp[x] == inf)
		{
			cout << -1 << endl; // El PDF imprime 1, pero -1 o "IMPOSSIBLE" es más estándar
		} else {
			cout << dp [x] << endl;
		}
		
		return 0;
	}
\end{lstlisting}