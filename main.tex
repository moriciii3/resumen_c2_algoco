\documentclass[10pt, a4paper]{article} 
\usepackage[utf8]{inputenc}
\usepackage[T1]{fontenc} % Mejor manejo de fuentes y guiones
\usepackage[spanish]{babel}
\usepackage[margin=1in]{geometry}
\usepackage{amsmath}
\usepackage{amssymb}
\usepackage{listings}
% --- MODIFICACIÓN CLAVE DE HYPERREF ---
\usepackage[
colorlinks=true, % Usa color en lugar de cuadros
linkcolor=blue,  % Color para enlaces internos (índice a secciones)
urlcolor=blue,   % Color para URLs
pdfborder={0 0 0} % Esto asegura que no haya borde en el PDF
]{hyperref} 
% ---------------------------------------
\usepackage{xcolor} % <-- CORRECCIÓN: Paquete para definir colores
\usepackage{tocloft} % Para mejor control del índice (opcional, pero ayuda)

% --- Configuración para Listings (Presentación de Código) ---
\definecolor{codegray}{rgb}{0.5,0.5,0.5} % <-- CORRECCIÓN: Define colores
\definecolor{codepurple}{rgb}{0.58,0,0.82} % <-- CORRECCIÓN: Define colores
\definecolor{backcolour}{rgb}{0.98,0.98,0.98} % <-- CORRECCIÓN: Define colores

\lstdefinestyle{mystyle}{ % <-- CORRECCIÓN: Define estilo
	backgroundcolor=\color{backcolour},   
	commentstyle=\color{codegray},
	keywordstyle=\color{blue},
	numberstyle=\tiny\color{codegray},
	stringstyle=\color{codepurple},
	basicstyle=\footnotesize\ttfamily,
	breakatwhitespace=false,         
	breaklines=true,                 
	captionpos=b,                    
	keepspaces=true,                 
	numbers=left,                    
	numbersep=5pt,                  
	showspaces=false,                
	showstringspaces=false,
	showtabs=false,                  
	tabsize=2,
	language=C++
}
\lstset{style=mystyle}

% --- Comando para importar ejercicios ---
\newcommand{\ejercicio}[2]{
	\section*{Ejercicio: #1}
	\addcontentsline{toc}{subsection}{#1} % Añade subsección al índice
	\input{#2}
	\vspace{0.5cm} 
}

\title{\textbf{Resumen de Algoritmos y Complejidad}} % Título sin emoji
\author{Apuntes para la Prueba}
\date{\today}

\begin{document}
	
	\maketitle
	\thispagestyle{empty} 
	
	% --- Índice ---
	\newpage
	\tableofcontents
	\thispagestyle{empty}
	\newpage
	
	% -------------------------------------------------------------
	% 1) Plantilla base y funciones comunes
	% -------------------------------------------------------------
	\section{Base}
	\addcontentsline{toc}{section}{Plantilla Base y Funciones Comunes}
	
	\ejercicio{Plantilla Base C++ para Competitiva/Pruebas}{base/plantilla}
	
	% -------------------------------------------------------------
	% 2) Algoritmos Greedy
	% -------------------------------------------------------------
	\section{Greedy}
	\addcontentsline{toc}{section}{Algoritmos Greedy}
	
	\ejercicio{Algoritmo de Kruskal (MST)}{greedy/kruskal}
	\ejercicio{Algoritmo de Kruskal 2 (MST)}{greedy/kruskal2}
	\ejercicio{Mínima Distancia para Iluminar (Faroles)}{greedy/faroles}
	\ejercicio{Construir Número Máximo con Presupuesto (Cercas)}{greedy/paint\_digits}
	
	% -------------------------------------------------------------
	% 3) Técnicas y Estructuras Comunes
	% -------------------------------------------------------------
	\section{Técnicas}
	\addcontentsline{toc}{section}{Técnicas y Estructuras Comunes}
	
	\ejercicio{Subarreglos con Suma Objetivo (Sliding Window)}{common\_techniques/sliding\_window\_sum}
	\ejercicio{Construcción de Palíndromos}{common\_techniques/palindrome\_construction}
	\ejercicio{Algoritmo de Bellman-Ford (Caminos Mínimos, Fuente Única)}{common_techniques/bellman_ford} % <-- NUEVO
	\ejercicio{Algoritmo de Floyd-Warshall (Caminos Mínimos, Todos los Pares)}{common_techniques/floyd_warshall} % <-- NUEVO
	
	% -------------------------------------------------------------
	% 4) Backtracking
	% -------------------------------------------------------------
	\section{Backtracking}
	\addcontentsline{toc}{section}{Algoritmos Backtracking}
	
	\ejercicio{Siguiente Permutación Lexicográfica (Usando next\_permutation)}{backtracking/next\_permutation}
	\ejercicio{Generación de Todas las Permutaciones}{backtracking/all\_permutations}
	
	% -------------------------------------------------------------
	% 5) Programación Dinámica (DP)
	% -------------------------------------------------------------
	\section{DP}
	\addcontentsline{toc}{section}{Programación Dinámica (DP)}
	
	\ejercicio{Problema de la Mochila (Knapsack 0/1) con Reconstrucción}{dp/knapsack}
	\ejercicio{Maximum Weight Independent Set (MWIS) en Grafo de Camino}{dp/mwis}
	\ejercicio{Cantidad Mínima de Monedas (Coin Change Minimum)}{dp/coin\_change\_min}
	\ejercicio{Combinaciones de Monedas (El Orden Importa)}{dp/coin\_change\_combinations\_order\_matters}
	\ejercicio{Combinaciones de Monedas (El Orden No Importa)}{dp/coin\_change\_combinations\_order\_doesnt\_matter}
	\ejercicio{Combinaciones con Reconstrucción Única (Problema del Camarero)}{dp/unique\_reconstruction}
	\ejercicio{Manejo de Entrada Mixta (getline y stringstream)}{dp/getline\_example}
	
	\end{document}