% greedy/faroles.tex

\subsubsection{Mínima Distancia para Iluminar (Faroles)}

El problema busca la mínima distancia (radio $d$) para que los faroles, cuyas posiciones están dadas, iluminen completamente una calle de longitud $L$. La solución se basa en encontrar la máxima distancia entre faroles adyacentes y las distancias a los extremos (0 y $L$) después de ordenar las posiciones.

\paragraph{Complejidad:}
* \textbf{Tiempo:} $O(n \log n)$ (por la ordenación).
* \textbf{Espacio:} $O(n)$.

% --- Código C++ ---
\subsubsection{Código C++ (Faroles)}
\begin{lstlisting}[
	caption={Mínima distancia requerida para iluminar una calle},
	extendedchars=true, 
	literate={ñ}{{\~n}}1 {Ñ}{{\~N}}1 {á}{{\'a}}1 {é}{{\'e}}1 {í}{{\'i}}1 {ó}{{\'o}}1 {ú}{{\'u}}1 {Á}{{\'A}}1 {É}{{\'E}}1 {Í}{{\'I}}1 {Ó}{{\'O}}1 {Ú}{{\'U}}1
	]
	#include <bits/stdc++.h>
	using namespace std;
	using 11 long long;
	
	int main() {
		11 n, 1;
		cin >> n >> 1;
		vector<double> pos(n);
		for (ll i=0; i < n; i++) {
			cin >> pos[i];
		}
		sort(pos.begin(), pos.end());
		double d=0;
		for (11 i=0; i<n - 1; i++) {
			d=max(d, (pos[i + 1] - pos[i]) / 2);
		}
		d=max(d, pos[0]);
		d=max(d, 1 - pos[n-1]);
		cout << fixed << setprecision (10) << d << endl;
		return 0;
	}
\end{lstlisting}